% Datei: vorlesung.tex
% LaTeX-Vorlage für Lineare Algebra (Deutsch, A4)
\documentclass[a4paper,11pt,oneside]{scrartcl}

% Sprache & Encoding
\usepackage[ngerman]{babel}
\usepackage[utf8]{inputenc}
\usepackage[T1]{fontenc}

% Schrift & Mikrotypographie
\usepackage{lmodern}
\usepackage{microtype}

% Seitenränder
\usepackage[
    left=25mm,
    right=25mm,
    top=25mm,
    bottom=25mm
]{geometry}

% Mathematik
\usepackage{amsmath,amssymb,amsthm,mathtools}
\usepackage{bm}            % fette Mathe-Symbole
\usepackage{physics}       % \norm, \abs, \eval, \ket, etc. (optional)
\allowdisplaybreaks

% Farben, Links
\usepackage{xcolor}
\usepackage[hidelinks]{hyperref}

% Nützliche Operatoren und Kurzbefehle
\newcommand{\trans}{^{\mathsf{T}}} % Transponieren
\newcommand{\mat}[1]{\begin{pmatrix} #1 \end{pmatrix}}

% Theorem-Umgebungen (Deutsch)
\theoremstyle{plain}
\newtheorem{satz}{Satz}[section]
\newtheorem{lemma}[satz]{Lemma}
\newtheorem{korollar}[satz]{Korollar}
\newtheorem{proposition}[satz]{Proposition}

\theoremstyle{definition}
\newtheorem{definition}[satz]{Definition}
\newtheorem{beispiel}[satz]{Beispiel}
\newtheorem{bemerkung}[satz]{Bemerkung}
\newtheorem{notation}[satz]{Notation}

% Anpassen von Beweis-Umgebung (Beweis -> Beweis)
\renewcommand{\proofname}{Beweis}

% Kopf- und Fußzeilen (optional)
\usepackage{scrlayer-scrpage}
\clearpairofpagestyles
\ihead{Lineare Algebra}
\ohead{\pagemark}
\cfoot{}

% Dokumentinformationen
\title{Lineare Algebra --- Vorlesungsnotizen}
\date{WiSe 2025/2026}

\begin{document}
\pagenumbering{gobble}
\maketitle
\newpage
\pagenumbering{roman}
\setcounter{page}{1}
\tableofcontents
\newpage
\bigskip
\pagenumbering{arabic}
\setcounter{page}{1}

\section{Einführung}
Kurz notieren: Ziel, Aufbau der Vorlesung, empfohlene Literatur.

\section{Vektorräume und lineare Abbildungen}
\begin{definition}
Ein \emph{Vektorraum} über dem Körper \(K\) ist eine Menge \(V\) mit ...
\end{definition}

\begin{satz}[Basis und Dimension]
Sei \(V\) ein endlichdimensionaler Vektorraum. Dann hat jede Basis dieselbe Anzahl an Elementen, die Dimension heißt \(\dim V\).
\end{satz}

\begin{proof}
Skizze des Beweises...
\end{proof}

\begin{beispiel}
Standardbasis des $\mathbb{R^n}$: $e_1,\dots,e_n$.
\end{beispiel}

\section{Matrizen und Determinanten}
Definitionen, Rechenregeln, Zusammenhang zu linearen Abbildungen.

\section{Eigenwerte und Eigenvektoren}
Definition von Eigenwerten, Diagonalisierbarkeit, Jordan-Normalform (falls relevant).

\section{Weitere Themen}
Skalarprodukte, orthogonale Projektionen, Ränge, Singulärwertzerlegung, Anwendungen.

% Ende des Dokuments
\end{document}
