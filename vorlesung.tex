% Datei: vorlesung.tex
% LaTeX-Vorlage für Lineare Algebra (Deutsch, A4)
\documentclass[a4paper,11pt,oneside]{scrartcl}

% Sprache & Encoding
\usepackage[ngerman]{babel}
\usepackage[utf8]{inputenc}
\usepackage[T1]{fontenc}

% Schrift & Mikrotypographie
\usepackage{lmodern}
\usepackage{microtype}

% Seitenränder
\usepackage[
    left=25mm,
    right=25mm,
    top=25mm,
    bottom=25mm
]{geometry}

% Mathematik
\usepackage{amsmath,amssymb,amsthm,mathtools}
\usepackage{bm}            % fette Mathe-Symbole
\usepackage{physics}       % \norm, \abs, \eval, \ket, etc. (optional)
\allowdisplaybreaks

% Farben, Links
\usepackage{xcolor}
\usepackage[hidelinks]{hyperref}

% Nützliche Operatoren und Kurzbefehle
\newcommand{\trans}{^{\mathsf{T}}} % Transponieren
\newcommand{\mat}[1]{\begin{pmatrix} #1 \end{pmatrix}}

% Theorem-Umgebungen (Deutsch)
\theoremstyle{plain}
\newtheorem{satz}{Satz}[section]
\newtheorem{lemma}[satz]{Lemma}
\newtheorem{korollar}[satz]{Korollar}
\newtheorem{proposition}[satz]{Proposition}

\theoremstyle{definition}
\newtheorem{definition}[satz]{Definition}
\newtheorem{beispiel}[satz]{Beispiel}
\newtheorem{bemerkung}[satz]{Bemerkung}
\newtheorem{notation}[satz]{Notation}

% Anpassen von Beweis-Umgebung (Beweis -> Beweis)
\renewcommand{\proofname}{Beweis}

% Kopf- und Fußzeilen (optional)
\usepackage{scrlayer-scrpage}
\clearpairofpagestyles
\ihead{Lineare Algebra}
\ohead{\pagemark}
\cfoot{}

% Dokumentinformationen
\title{Lineare Algebra --- Vorlesungsnotizen}
\date{WiSe 2025/2026}

\begin{document}
\pagenumbering{gobble}
\maketitle
\newpage
\pagenumbering{roman}
\setcounter{page}{1}
\tableofcontents
\newpage
\bigskip
\pagenumbering{arabic}
\setcounter{page}{1}


\section{Notation}
$X,Y$ Aussage \\
$X \land Y$: $X$ und $Y$ \\
$X \lor Y$: $X$ oder $Y$ \\
$\lnot X$: nicht $X$ (Negation) \\
$X \Rightarrow Y$: aus $X$ folgt $Y$ \\
$X \Leftrightarrow Y$: $X$ ist äquivalent zu $Y$ \\
$\forall x \in M: P(x)$: für alle $x$ in $M$ gilt $P(x)$ \\
$\exists x \in M: P(x)$: es existiert ein $x$ in $M$ mit $P(x)$ \\

\section{Mengen}
\begin{definition}
    \item Menge wird durch Angabe ihrer Elemente definiert.
    \item Zwei Mengen sind genau dann gleich, wenn sie dieselben Elemente haben.
\end{definition}
\begin{notation}
    \item $m \in M$: $\Leftrightarrow$ $m$ ist ein Element von $M$.
    \item $m \notin M$: $\Leftrightarrow$ $m$ ist kein Element von $M$.
    \item Beispiel: $M = \{1,2,3\}$, dann ist $1 \in M$ und $4 \notin M$.
    \item $\emptyset = \{\}$ (leere Menge): Menge ohne Elemente.
    \item $\mathbb{N}$ = natürliche Zahlen = $\{1,2,3,\ldots\}$
    \item $\mathbb{N}_0$ = natürliche Zahlen mit Null = $\{0,1,2,3,\ldots\}$
    \item $\mathbb{Z}$ = ganze Zahlen = $\{\ldots,-2,-1,0,1,2,\ldots\}$
    \item $\mathbb{Q}$ = rationale Zahlen = $\left\{\frac{p}{q} \mid p \in \mathbb{Z} \land q \in \mathbb{Z} \land q \neq 0 \right\}$
    \item $\mathbb{R}$ = reelle Zahlen
    \item $\mathbb{C}$ = komplexe Zahlen
\end{notation}
\begin{definition}
    \item Eine Teilmenge $N$ einer Menge $M$ ist eine Menge, deren Elemente allesamt auch in $M$ liegen.
    \item Schreibweise: $N \subseteq M$.
    \item $$N \subseteq M \Leftrightarrow \forall n \in N: n \in M.$$
    \item $$N \nsubseteq M \Leftrightarrow \exists n \in N: n \notin M.$$
    \item $$N \subsetneqq M \Leftrightarrow N \subseteq M \land N \neq M.$$
    \item Beispiel: $\emptyset \subseteq \mathbb{N} \subseteq \mathbb{N}_0 \subseteq \mathbb{Z} \subseteq \mathbb{Q} \subseteq \mathbb{R} \subseteq \mathbb{C}$
\end{definition}
\begin{notation}
    \item Zwei Mengen sind genau dann gleich, wenn sie wechselseitig Teilmengen voneinander sind:
    \item $$N = M \Leftrightarrow N \subseteq M \land M \subseteq N.$$
\end{notation}

% Ende des Dokuments
\end{document}
